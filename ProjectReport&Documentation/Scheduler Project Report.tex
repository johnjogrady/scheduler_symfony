\documentclass[a4paper]{article}

\usepackage{graphicx}	
\begin{document}
\begin{titlepage}

\newcommand{\HRule}{\rule{\linewidth}{0.5mm}}

\title {Carers Scheduling Prototype Software Application}

\author{John O'Grady}
\date{April 2017}

\maketitle
\HRule \\[0.4cm]
\thispagestyle{empty}

\begin{center}
\textsc{\large Submitted in part fulfilment
 \\for the degree of
Higher Diploma in Computing
\\School of Informatics and Engineering,
\\Institute of Technology Blanchardstown,
\\Dublin, Ireland
} 
\end{center}

\HRule \\[2cm]

%----------------------------------------------------------------------------------------
%	AUTHOR SECTION
%----------------------------------------------------------------------------------------

\begin{minipage}{0.4\textwidth}
\begin{flushleft} \large
\emph{Author:}\\
 \textsc{John O'Grady} 
\end{flushleft}
\end{minipage}
~
\begin{minipage}{0.4\textwidth}
\begin{flushright} \large
\emph{Supervisor:} \\
\textsc{Dr. Matt Smith} 
\end{flushright}
\end{minipage}\\[2cm]

\begin{center}
\includegraphics{itb_logo.jpg}
\end{center}

\end{titlepage}
\pagenumbering{arabic}
%----------------------------------------------------------------------------------------
%	Brief overview of project, aims, methodology and approach taken
%----------------------------------------------------------------------------------------
\section*{Abstract}

\pagebreak
\tableofcontents
\pagebreak------------------------------------------------
\section{Introduction}

\section{Organisational Context}
%----------------------------------------------------------------------------------------
%	Background to work of IWA and context in which app will be used
%       Reference and link to live tender and explanation of tender approach
%       Description of service and roles who will use the live applciation
%	Definition and scope of prototype application
%----------------------------------------------------------------------------------------

\section {Literature Review}

%----------------------------------------------------------------------------------------
%	Overview of fields reviewed and sources consulted
%       Review of SDLC, Agile vs Waterfall
%       UML Methodology- critical evaluation of UML
%       User Testing Approaches
%       MVC and alternative frameworks- why I chose MVC
% 	The Symfony Framework [and alternatives]
%----------------------------------------


\section {Requirements Analysis}

%----------------------------------------------------------------------------------------
%	Review of open source and commercial applications in similar functionality sphereRequirements specification
%       Chapter 4: 	System Requirements and Specification
\iffalse
%Requirements Specification (Analysis)
Description / Usage:  The requirements specification lays out the detailed functional requirements for an application so that an
accurate depiction of the system is made in a way that holds both DAS-ITE and the customer accountable for the functionality
that must be delivered.  This document is also used to start building a test plan and, possibly, test scripts.
Criteria for Usage: Required for all projects
Example: TBD
%High Level Use Case Specification (Analysis)
Description / Usage:  The high-level use case specification aims to identify all the actors who will use a system and what
actions they can take in that system.
Criteria for Usage: Required for all projects.
Example: TBD
%Detailed Use Cases (Analysis)
Description / Usage:  The detailed use case document takes all identified use cases and establish when what triggers the
beginning of the use case, what the interaction is during the use case, and what action ends the use case.  This simply takes what
was developed by the high-level use cases and establishes more detail.
Criteria for Usage: Optional.  Detailed use cases make sense on high profile projects or for situations where a use case is
quite complicated.
Example: TBD
%High Level Design (Analysis)
Description / Usage:  The high-level design takes into consideration all the information provided by the preceding documents
and attempts to lay out an initial design.  This design identifies any design constraints, establishes a basic architecture as well as
the solution design which consists of any data models, data flow diagrams, application flow and initial screen mock-ups.
Criteria for Usage: Required for all projects.
Example: TBD
%Test Plan (Analysis, Design)
Description / Usage:  The test plan outlines how an application will be tested. The most substantial portions of the test plan
come from the requirements specification developed during the analysis phase.  However, during the design phase, additional
testing needs may surface.
Criteria for Usage: Required for every project
Example: TBD
%Test Script (Design, Construction)
Description / Usage:  The test script outlines the exact steps for performing a test.  This document tracks who is doing the
testing, what is being tested, how to set up testing, how to execute the test and whether or not the tests were successful or not.
Criteria for Usage: Optional
Example: TBD
\fi
%----------------------------------------
% Design artefacts- diagram of project structure UML, Database tables, Use cases
% Development includes how the project was developed from coding to database tables etc
% Implementation project plan, GANNT chart, dependencies, project diary as appendix [readme on Github commits?]
\end{document}