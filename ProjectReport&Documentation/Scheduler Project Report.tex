\documentclass[a4paper]{article}

\usepackage{graphicx}	
\begin{document}
\begin{titlepage}

\newcommand{\HRule}{\rule{\linewidth}{0.5mm}}

\title {Carers Scheduling Prototype Software Application}

\author{John O'Grady}
\date{April 2017}

\maketitle
\HRule \\[0.4cm]
\thispagestyle{empty}

\begin{center}
\textsc{\large Submitted in part fulfilment
 \\for the degree of
Higher Diploma in Computing
\\School of Informatics and Engineering,
\\Institute of Technology Blanchardstown,
\\Dublin, Ireland
} 
\end{center}

\HRule \\[2cm]

%----------------------------------------------------------------------------------------
%	AUTHOR SECTION
%----------------------------------------------------------------------------------------

\begin{minipage}{0.4\textwidth}
\begin{flushleft} \large
\emph{Author:}\\
 \textsc{John O'Grady} 
\end{flushleft}
\end{minipage}
~
\begin{minipage}{0.4\textwidth}
\begin{flushright} \large
\emph{Supervisor:} \\
\textsc{Dr. Matt Smith} 
\end{flushright}
\end{minipage}\\[2cm]

\begin{center}
\includegraphics{itb_logo.jpg}
\end{center}

\end{titlepage}
\pagenumbering{arabic}
%----------------------------------------------------------------------------------------
%	Brief overview of project, aims, methodology and approach taken
%----------------------------------------------------------------------------------------
\section*{Abstract}

\pagebreak
\tableofcontents
\pagebreak------------------------------------------------
\section{Introduction}

\section{Organisational Context}
%----------------------------------------------------------------------------------------
%	Background to work of IWA and context in which app will be used
%       Reference and link to live tender and explanation of tender approach
%       Description of service and roles who will use the live applciation
%	Definition and scope of prototype application
%----------------------------------------------------------------------------------------

\section {Literature Review}

%----------------------------------------------------------------------------------------
%	Overview of fields reviewed and sources consulted
%       Review of SDLC, Agile vs Waterfall
%       UML Methodology- critical evaluation of UML
%       User Testing Approaches
%       MVC and alternative frameworks- why I chose MVC
% 	The Symfony Framework [and alternatives]
%----------------------------------------

\section {Initiation}

% High level concept description

\section {Concept Development}


\subsection {Overview of Project Scope}
\subsection {Feasibility Review}
Review of open source and commercial applications in similar functionality sphereRequirements specification

\section {Planning Phase}

\subsection {Project Management}
\subsection {Risk Management}
\subsection {Communication Management}
\subsection {Stakeholder Management}
\section {Requirements Analysis}

\subsection {Functional Requirements Specification}

The requirements specification lays out the detailed functional requirements for an application so that an
accurate depiction of the system is made in a way that holds both the IT project lead and the customer accountable for the functionality that must be delivered.  This document is also used to start building a test plan and, possibly, test scripts.

\subsection {Use case specification}
The high-level use case specification aims to identify all the actors who will use a system and what
actions they can take in that system.

\subsection {Detailed Use Case Analysis}
The detailed use case document takes all identified use cases and establish when what triggers the
beginning of the us case, what the interaction is during the use case, and what action ends the use case.  This simply takes what
was developed by the high-level use cases and establishes more detail

\section {Design Specification}
\subsection {System Design Document}

The high-level design takes into consideration all the information provided by the preceding documents
and attempts to lay out an initial design.  This design identifies any design constraints, establishes a basic architecture as well as
the solution design which consists of any data models, data flow diagrams, application flow and initial screen mock-ups.

% Design artefacts- diagram of project structure UML, Database tables, Use cases
% Development includes how the project was developed from coding to database tables etc
% Implementation project plan, GANNT chart, dependencies, project diary as appendix [readme on Github commits?]



\section {Design Specification}
\subsection {Software Development Document}

The high-level design takes into consideration all the information provided by the preceding documents
and attempts to lay out an initial design.  This design identifies any design constraints, establishes a basic architecture as well as
the solution design which consists of any data models, data flow diagrams, application flow and initial screen mock-ups.
\subsection {Test Analysis Report}

\section {Testing Specification}
\subsection {Testing Approaches Used}
\subsection {Test Problem report}

\section {Implementation}
\subsection {Implementation Plan}

\section {Evaluation}
\subsection {User Feedback}
\subsection {Further Enhancements}

\section {Conclusions}
\subsection {Review of material covered }
\subsection {Further Development Opportunities}
\subsection {Outstanding Issues/Continuous Improvement Plan}

\section {Appendices}
\subsection {Code Listing}

\subsubsection {Database creation scripts}
\subsubsection {Entity Classes}
\subsubsection {Mapping Extension Classes}
\subsubsection {Repository Classes}
\subsubsection {Form Type}
\subsubsection {Controller Classes}
\subsubsection {Twig Templates}
\subsubsection {List of Vendor tools used}
\end{document}