\documentclass[a4paper,12pt]{article}

\usepackage{graphicx}	
\usepackage{csquotes}
\begin{document}
\begin{titlepage}

\newcommand{\HRule}{\rule{\linewidth}{0.5mm}}

\title {Carers Scheduling Prototype Software Application}

\author{John O'Grady}
\date{April 2017}

\maketitle

\thispagestyle{empty}

\begin{center}
\HRule \\[0.4cm]
\textsc{\large Submitted in part fulfilment
 \\for the degree of
Higher Diploma in Computing
\\School of Informatics and Engineering,
\\Institute of Technology Blanchardstown,
\\Dublin, Ireland
} 
\end{center}

\HRule \\[2cm]

%----------------------------------------------------------------------------------------
%	AUTHOR SECTION
%----------------------------------------------------------------------------------------

\begin{minipage}{0.4\textwidth}
\begin{flushleft} \large
\emph{Author:}\\
 \textsc{John O'Grady} 
\end{flushleft}
\end{minipage}
~
\begin{minipage}{0.4\textwidth}
\begin{flushright} \large
\emph{Supervisor:} \\
\textsc{Dr. Matt Smith} 
\end{flushright}
\end{minipage}\\[2cm]

\begin{center}
\includegraphics{itb_logo.jpg}
\end{center}

\end{titlepage}
\pagenumbering{arabic}
%----------------------------------------------------------------------------------------
%	Brief overview of project, aims, methodology and approach taken
%----------------------------------------------------------------------------------------
\section*{Abstract}

This project is concerned with requirements gathering, planning and delivery of a prototype rostering and scheduling software project for Irish Wheelchair Association (IWA), a large non profit service provider which needs a new organisation wide application to manage the planning of personal care delivery for people with disabilities. This project begins by examining the body of current academic research in software delivery to provide a brief synthesis of best practice in the planning, implementation and deployment of modern software applications.

Based on this research, this author selects the Systems Development Life Cycle (SDLC) methodology as an appropriate toolset to determine the requirements of IWA for a new rostering and scheduling solution which the organisation plans to deploy to its 1,500 Personal Assistants. Having gathered the requirements, the project progresses to define the organisational context for the software deployment.

Using the SDLC methodology the author, who is responsible for leading the Information Technology function at IWA, works collaboratively with key project stakeholders across the organisation to define a broad functional requirement specification and produces various key design artefacts.

As the development work on the project progresses, a suite of SDLC mandated project management documents are iteratively defined and a high level test plan is developed and implemented to ensure code quality and alignment with user expectations and requirements.

Finally, in conclusion, the author offers some insights from his experience of the development process together with consideration of some areas for further investigation and ongoing improvement of the prototype which are outside the scope of the initial project build.

\pagebreak
\tableofcontents
\pagebreak------------------------------------------------
\section{Introduction}

This project examines the requirements for a prototype Rostering and Scheduling solution for which the ultimate client, the Irish Wheelchar Association (IWA) has a real world business requirement. It is envisaged that the finished software deliverable at the conclusion of this project will not be a fully enterprise ready software solution but rather the project deliverable will be employed as a prototype for user acceptance testing by representative end users  and this process will be used to examine and further refine IWA's existing assumptions in relation to the imminent deployment of a ciritical real life deployment whose successful implementation is considered to be fundamental to the competitive position of the IWA. 

The scope of this project is to establish the detailed software functionality requirements and to utilise this information to deliver a prototype software solution for evaluation purposes which will cover the primary use cases and functional requirements identified. This will enable end users to test, assess and give feedback on the prototype and it is envisaged that this process will be constructive in mitigating the risk of significant functionality components being omitted from the final production system or the usability of the ultimate solution not meeting user requirements in full. Through this iterative process, it is planned to document a more comprehensive set of final functional requirements for the production deliverable will fundamentally underpin a competitive tender exercise and vendor engagement through which IWA will select, configure and implement a live rostering and scheduling solution for its 1,500 Personal Assistants across Ireland.


\section{Organisational Context}
\subsection {History of Irish Wheelchair Association}
The IWA is an independent organisation which was founded in 1960 by a group of people with disabilities and which is governed by a Board of Directors elected from its 20,000 membership base. It is the largest provider of Assisted Living Services in the Ireland, employing 2,500 employees and delivering over 2 million hours of service annually. It is substantially funded by the Irish Government, primarily through the Health Service Executive, from whom it receives funding as a Section 39 agency under the Health Act 2004, although it also receives funding from various other statutory and non statutory sources and raises funds directly from the general public. A challenge which is directly pertinent to the planned software project is that IWA's funding streams are primarily remitted in respect of delivery of front line services and it receives no direct funding for Information Technology or other Shared Services activities.

During the recent economic downturn and its impact on government funding, like many non service providers, IWA has had substantial cuts in its funding but it has nevertheless managed to maintain and in some cases grow its level of service activity. These funding challenges have impacted on IWA's Information Technology infrastructure which have suffered from historic underinvestment but IWA's strategic plan for the years 2017 to 2010 now positions Information Technology initiatives as a core enabler of driving efficiency in its service delivery, on the strict understanding that all investment decisions must linked to a clear business case which will deliver an identifiable and measurable return on investment to the organisation.

IWA has been at the forefront of developing person centred  service provision in Ireland, based on international best practice and it is unique among large charities in Ireland in that IWA remains wholly owned by and accountable to its 20,000 members who are people with disabilities and volunteers and other supporters. It advocates for independence and quality of life for all people with disabilities in Ireland. In this regard, the mission of the Irish Wheelchair has recently been updated in its new Strategic Plan for 2017-2020 as follows:
\begin{displayquote}
Irish Wheelchair Association  has a vision of an Ireland where people with disabilities enjoy equal rights, choices and opportunities in how they live their lives, and where our country is a model worldwide for a truly inclusive society.
\end{displayquote}

In addition to Assisted Living Service,


\subsection {Overview of Assisted Living Service}
\subsection {Current Position}
\subsection {Overview of Application Impacts}
%----------------------------------------------------------------------------------------
%	Background to work of IWA and context in which app will be used
%       Reference and link to live tender and explanation of tender approach
%       Description of service and roles who will use the live applciation
%	Definition and scope of prototype application
%----------------------------------------------------------------------------------------

\section {Literature Review}
\subsection {Software Development Paradigms}
\subsection {Review of the Systems Development Life Cycle Methodology}
\subsection {Complementary Methodologies of Development}

\subsubsection {Joint applications development (JAD)}
\subsubsection {Rapid application development (RAD)}
\subsubsection {Extreme programming (XP)}
\subsubsection {Open-source development}
\subsubsection {End-user development}
\subsubsection {Object-oriented programming}

\subsection {Agile vs. Waterfall}
\subsubsection {Evolution of Agile}
\subsubsection {Evolution of Waterfall}
\subsubsection {Critical Comparatives}
\subsection {Unified Modelling Language}
\subsection {Advantages of Unified Modelling Language}
\subsection {Advantages of Unified Modelling Language}


%----------------------------------------------------------------------------------------
%	Overview of fields reviewed and sources consulted
%       Review of SDLC, Agile vs Waterfall
%       UML Methodology- critical evaluation of UML
%       User Testing Approaches
%       MVC and alternative frameworks- why I chose MVC
% 	The Symfony Framework [and alternatives]
%----------------------------------------

\section {Initiation}

% High level concept description

\section {Concept Development}


\subsection {Overview of Project Scope}
\subsection {Feasibility Review}
Review of open source and commercial applications in similar functionality sphereRequirements specification

\section {Planning Phase}

\subsection {Project Management}
\subsection {Risk Management}
\subsection {Communication Management}
\subsection {Stakeholder Management}
\section {Requirements Analysis}

\subsection {Functional Requirements Specification}

The requirements specification lays out the detailed functional requirements for an application so that an
accurate depiction of the system is made in a way that holds both the IT project lead and the customer accountable for the functionality that must be delivered.  This document is also used to start building a test plan and, possibly, test scripts.

\subsection {Use case specification}
The high-level use case specification aims to identify all the actors who will use a system and what
actions they can take in that system.

\subsection {Detailed Use Case Analysis}
The detailed use case document takes all identified use cases and establish when what triggers the
beginning of the us case, what the interaction is during the use case, and what action ends the use case.  This simply takes what
was developed by the high-level use cases and establishes more detail

\section {Design Specification}
\subsection {System Design Document}

The high-level design takes into consideration all the information provided by the preceding documents
and attempts to lay out an initial design.  This design identifies any design constraints, establishes a basic architecture as well as
the solution design which consists of any data models, data flow diagrams, application flow and initial screen mock-ups.

% Design artefacts- diagram of project structure UML, Database tables, Use cases
% Development includes how the project was developed from coding to database tables etc
% Implementation project plan, GANNT chart, dependencies, project diary as appendix [readme on Github commits?]



\section {Design Specification}
\subsection {Software Development Document}

The high-level design takes into consideration all the information provided by the preceding documents
and attempts to lay out an initial design.  This design identifies any design constraints, establishes a basic architecture as well as
the solution design which consists of any data models, data flow diagrams, application flow and initial screen mock-ups.
\subsection {Test Analysis Report}

\section {Testing Specification}
\subsection {Testing Approaches Used}
\subsection {Test Problem report}

\section {Implementation}
\subsection {Implementation Plan}

\section {Evaluation}
\subsection {User Feedback}
\subsection {Further Enhancements}

\section {Conclusions}
\subsection {Review of material covered }
\subsection {Further Development Opportunities}
\subsection {Outstanding Issues/Continuous Improvement Plan}

\section {Appendices}
\subsection {Code Listing}

\subsubsection {Database creation scripts}
\subsubsection {Entity Classes}
\subsubsection {Mapping Extension Classes}
\subsubsection {Repository Classes}
\subsubsection {Form Type}
\subsubsection {Controller Classes}
\subsubsection {Twig Templates}
\subsubsection {List of Vendor tools used}
\end{document}